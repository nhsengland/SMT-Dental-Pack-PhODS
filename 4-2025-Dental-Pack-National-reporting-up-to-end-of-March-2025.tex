% Options for packages loaded elsewhere
\PassOptionsToPackage{unicode}{hyperref}
\PassOptionsToPackage{hyphens}{url}
%
\documentclass[
  8pt,
  ignorenonframetext,
  aspectratio = 169]{beamer}
\usepackage{pgfpages}
\setbeamertemplate{caption}[numbered]
\setbeamertemplate{caption label separator}{: }
\setbeamercolor{caption name}{fg=normal text.fg}
\beamertemplatenavigationsymbolsempty
% Prevent slide breaks in the middle of a paragraph
\widowpenalties 1 10000
\raggedbottom
\setbeamertemplate{part page}{
  \centering
  \begin{beamercolorbox}[sep=16pt,center]{part title}
    \usebeamerfont{part title}\insertpart\par
  \end{beamercolorbox}
}
\setbeamertemplate{section page}{
  \centering
  \begin{beamercolorbox}[sep=12pt,center]{part title}
    \usebeamerfont{section title}\insertsection\par
  \end{beamercolorbox}
}
\setbeamertemplate{subsection page}{
  \centering
  \begin{beamercolorbox}[sep=8pt,center]{part title}
    \usebeamerfont{subsection title}\insertsubsection\par
  \end{beamercolorbox}
}
\AtBeginPart{
  \frame{\partpage}
}
\AtBeginSection{
  \ifbibliography
  \else
    \frame{\sectionpage}
  \fi
}
\AtBeginSubsection{
  \frame{\subsectionpage}
}
\usepackage{amsmath,amssymb}
\usepackage{lmodern}
\usepackage{iftex}
\ifPDFTeX
  \usepackage[T1]{fontenc}
  \usepackage[utf8]{inputenc}
  \usepackage{textcomp} % provide euro and other symbols
\else % if luatex or xetex
  \usepackage{unicode-math}
  \defaultfontfeatures{Scale=MatchLowercase}
  \defaultfontfeatures[\rmfamily]{Ligatures=TeX,Scale=1}
  \setsansfont[]{Calibri Light}
\fi
% Use upquote if available, for straight quotes in verbatim environments
\IfFileExists{upquote.sty}{\usepackage{upquote}}{}
\IfFileExists{microtype.sty}{% use microtype if available
  \usepackage[]{microtype}
  \UseMicrotypeSet[protrusion]{basicmath} % disable protrusion for tt fonts
}{}
\makeatletter
\@ifundefined{KOMAClassName}{% if non-KOMA class
  \IfFileExists{parskip.sty}{%
    \usepackage{parskip}
  }{% else
    \setlength{\parindent}{0pt}
    \setlength{\parskip}{6pt plus 2pt minus 1pt}}
}{% if KOMA class
  \KOMAoptions{parskip=half}}
\makeatother
\usepackage{xcolor}
\newif\ifbibliography
\usepackage{graphicx}
\makeatletter
\def\maxwidth{\ifdim\Gin@nat@width>\linewidth\linewidth\else\Gin@nat@width\fi}
\def\maxheight{\ifdim\Gin@nat@height>\textheight\textheight\else\Gin@nat@height\fi}
\makeatother
% Scale images if necessary, so that they will not overflow the page
% margins by default, and it is still possible to overwrite the defaults
% using explicit options in \includegraphics[width, height, ...]{}
\setkeys{Gin}{width=\maxwidth,height=\maxheight,keepaspectratio}
% Set default figure placement to htbp
\makeatletter
\def\fps@figure{htbp}
\makeatother
\setlength{\emergencystretch}{3em} % prevent overfull lines
\providecommand{\tightlist}{%
  \setlength{\itemsep}{0pt}\setlength{\parskip}{0pt}}
\setcounter{secnumdepth}{-\maxdimen} % remove section numbering
\titlegraphic{\includegraphics[width=12cm, height=2cm]{footer.JPG}}
\setbeamertemplate{footline}
{
    \leavevmode%
    \hbox{%
        \begin{beamercolorbox}[wd = .5\paperwidth, ht = 1ex, dp = 1ex, center]{author in head/foot}%
            CONFIDENTIAL: Not for onward sharing
        \end{beamercolorbox}%
        \begin{beamercolorbox}[wd = .5\paperwidth, ht = 1ex, dp = 1ex, center]{date in head/foot}%
            \insertframenumber{}
        \end{beamercolorbox}
    }%
    \vskip3pt%
}

\setbeamertemplate{headline}
{
    \leavevmode%
    \hbox{%
        \begin{beamercolorbox}[wd = \paperwidth, ht = 20pt, dp = 0pt, right]{date in head/foot}%
            \includegraphics[width=1.5cm, height=4.5cm]{logo.JPG}
        \end{beamercolorbox}
    }%
    % \vskip3pt%
}

\addtobeamertemplate{frametitle}{\vspace*{-0.7cm}}{\vspace*{0.2cm}}



% \usepackage{graphicx}
% \usepackage{fancyhdr}
%    \pagestyle{fancy}
%    \fancyhead[R]{\includegraphics[width=1.5cm, height=4.5cm]{logo.JPG}}
% \renewcommand{\headrulewidth}{0pt}
% 
% \addtobeamertemplate{footline}{\begin{center}\rule{0.6\paperwidth}{0.4pt}\end{center}\vspace*{-1ex}}{}

% \setbeamertemplate{headline}
% {
%     \leavevmode%
%     \hbox{%
%         \begin{beamercolorbox}[wd = .5\paperwidth, ht = 1ex, dp = 1ex, center]{author in head/foot}%
%             \pgfdeclareimage[height=1.5cm, width=4.5cm]{logo}{logo.JPG}
%             \logo{\pgfuseimage{logo}}
%         \end{beamercolorbox}
%     }%
%     \vskip3pt%
% }
% 
% \addtobeamertemplate{headline}{\begin{center}\rule{0.6\paperwidth}{0.4pt}\end{center}\vspace*{-1ex}}{}

\ifLuaTeX
  \usepackage{selnolig}  % disable illegal ligatures
\fi
\IfFileExists{bookmark.sty}{\usepackage{bookmark}}{\usepackage{hyperref}}
\IfFileExists{xurl.sty}{\usepackage{xurl}}{} % add URL line breaks if available
\urlstyle{same} % disable monospaced font for URLs
\hypersetup{
  pdftitle={Dental Data Pack},
  pdfauthor={April, 2025 - Reporting up to end of April, 2025},
  hidelinks,
  pdfcreator={LaTeX via pandoc}}

\title{Dental Data Pack}
\subtitle{National}
\author{April, 2025 - Reporting up to end of April, 2025}
\date{}

\begin{document}
\frame{\titlepage}

\begin{frame}[allowframebreaks]
  \tableofcontents[hideallsubsections]
\end{frame}
\hypertarget{introduction}{%
\section{Introduction}\label{introduction}}

\begin{frame}{Introduction}
\protect\hypertarget{introduction-1}{}
This slide pack has been streamlined to help users to easily access the
most important information. This change took effect as of June 2024
data. As part of this change we have also sought to make the slides more
useful by making the following changes:

\begin{itemize}
\tightlist
\item
  Swapping from scheduled data (the month the dentist claimed for
  payment following a finished course of treatment) to calendar month
  data (the month the course of treatment finished). This change better
  reflects the activity of dentists.
\item
  Including only GDS/PDS/PDS+ contracts where total contracted UDAs
  \textgreater100. This means contracts with low UDA targets are
  excluded because they are usually not monitored against UDA delivery.
\item
  A national HTML will no longer be produced, instead the data is in an
  Excel file available alongside this pack.
\end{itemize}

~

This pack will continue to evolve and change, any questions/comments
should be directed to
\href{mailto:england.primarycare.phodsdata@nhs.net}{\nolinkurl{england.primarycare.phodsdata@nhs.net}}
\end{frame}

\begin{frame}{Calendar vs scheduled data}
\protect\hypertarget{calendar-vs-scheduled-data}{}
Analyses in this report now only use \textbf{calendar data}. Previous
packs used scheduled data. An explanation on the difference is given
below. The appendix also provides more detail on how the switch effects
the data.

\begin{block}{Calendar data explanation}
\protect\hypertarget{calendar-data-explanation}{}
\begin{itemize}
\tightlist
\item
  Calendar data represents the month that a Course of Treatment (CoT)
  was completed.
\item
  For calendar data, if a CoT was completed in February but not declared
  till March, that activity would still be recorded as occurring in
  February.
\end{itemize}
\end{block}

\begin{block}{Scheduled data explanation}
\protect\hypertarget{scheduled-data-explanation}{}
\begin{itemize}
\tightlist
\item
  Scheduled data represents the month that a Course of Treatment (CoT)
  was claimed for in the BSA Compass system.
\item
  For scheduled data, if a CoT was completed in February but not
  declared till March, the financial activity would recorded as
  occurring in March.
\end{itemize}

~

The swap to using calendar data allows us to more accurately see what
activity is happening per month rather than what payments are being
claimed per month.

~

Note that following a finished Course of Treatment (CoT), dentists have
62 days to declare their activity via FP17 forms to the BSA Compass
system. This means when using calendar data, the most recent two months
data are incomplete and subject to change. Therefore, the data that is
provisional will be identified via a dashed line where possible and data
labels on graphs will be for the most recent final month's data.
\end{block}
\end{frame}

\hypertarget{summary}{%
\section{Summary}\label{summary}}

\begin{frame}{Summary}
\protect\hypertarget{summary-1}{}
\begin{block}{Dental}
\protect\hypertarget{dental}{}
\begin{itemize}
\tightlist
\item
  February and March data is subject to change for dental data. January
  2025 data is referenced here in the summary.
\item
  Count of dental contracts included (January 2025): 6,333.
\item
  UDA delivery for most recent final month (January 2025): 5,906,479, a
  change of 862,040.6 from previous month (December 2024), standardised
  to allow for differing month lengths.
\item
  Year to date UDA delivery for most recent final month (January 2025):
  58,098,291 and as percentage of annual contracted UDAs: 84\%.
\item
  Unique adults seen in the last 24 months as at January 2025:
  18,184,675, a change of 417 from previous month (December 2024).
\item
  Unique children seen in the last 12 months as at January 2025:
  7,099,893, a change of 9,425 from previous month (December 2024).
\item
  Adults seen as part of New Patient Premium scheme as at January 2025:
  219,590, a change of 38,852 from previous month (December 2024).
\item
  Children seen as part of New Patient Premium scheme as at January
  2025: 123,862, a change of 17,826 from previous month (December 2024).
\end{itemize}
\end{block}

\begin{block}{Orthodontics}
\protect\hypertarget{orthodontics}{}
\begin{itemize}
\tightlist
\item
  Count of orthodontic contracts included (March 2025): 764.
\item
  Orthodontic activity is subject to change within a financial year. As
  of March 2025 year to date UOA delivery is 3,467,114.
\end{itemize}
\end{block}
\end{frame}

\hypertarget{units-of-dental-activity-uda-information}{%
\section{Units of Dental Activity (UDA)
Information}\label{units-of-dental-activity-uda-information}}

\begin{frame}{Units of Dental Activity - Calendar activity data
standardised by working days}
\protect\hypertarget{units-of-dental-activity---calendar-activity-data-standardised-by-working-days}{}
\begin{center}\includegraphics{4-2025-Dental-Pack-National-reporting-up-to-end-of-March-2025_files/figure-beamer/UDA_activity_Calendar2-1} \end{center}

\begin{itemize}
\tightlist
\item
  These graphs show the percentage of UDAs delivered compared to the
  expected delivery. Expected monthly delivery is standardised by
  working days in the month and calculated as Target UDAs* Monthly
  working days/working days per year.
\item
  For the current financial year, monthly performance is shown. For the
  previous financial years, annual average performance is shown.
\item
  Over the course of the financial year this graph allows us to see
  patterns of delivery.
\end{itemize}
\end{frame}

\begin{frame}{Units of Dental Activity - Calendar year to date delivery}
\protect\hypertarget{units-of-dental-activity---calendar-year-to-date-delivery}{}
\begin{center}\includegraphics{4-2025-Dental-Pack-National-reporting-up-to-end-of-March-2025_files/figure-beamer/YTD UDA-1} \end{center}

\begin{itemize}
\tightlist
\item
  This graph shows the progress towards delivering against contracted
  UDAs in the financial year. Here, expected delivery assumes delivery
  is equal across all months.
\end{itemize}
\end{frame}

\begin{frame}{Units of Dental Activity - Monthly activity compared to
previous financial year}
\protect\hypertarget{units-of-dental-activity---monthly-activity-compared-to-previous-financial-year}{}
\begin{center}\includegraphics{4-2025-Dental-Pack-National-reporting-up-to-end-of-March-2025_files/figure-beamer/monthly UDAs-1} \end{center}

\begin{itemize}
\tightlist
\item
  This graph shows the number of UDAs delivered each month for the
  current financial year compared with the previous financial year.
\item
  Some of the difference between months may be due to differences in the
  number of working days. The table shows the number of working days in
  each month for 2023/24 and 2024/25.
\item
  As dentists have 62 days to claim for their activity, the most recent
  two months data are incomplete. We expect final data to be higher.
  Figures for the latest complete month of data are presented on this
  graph.
\end{itemize}
\end{frame}

\begin{frame}{Banded courses of treatment standardised by working days
(including Foundation Dentists (FD))}
\protect\hypertarget{banded-courses-of-treatment-standardised-by-working-days-including-foundation-dentists-fd}{}
\begin{center}\includegraphics{4-2025-Dental-Pack-National-reporting-up-to-end-of-March-2025_files/figure-beamer/banded CoT 1-1} \end{center}

\begin{itemize}
\tightlist
\item
  This graph shows the number of completed Courses of Treatment (CoTs)
  by band over time. Data has been standardised to working days to
  remove variation due to month length. It is calculated as completed
  CoTs*Monthly working days/working days per year.
\item
  From November 2022 band 2 was split into 2a, 2b and 2c. More
  information can be viewed here:
  \href{https://www.nhsbsa.nhs.uk/units-dental-activity-uda-changes-band-2-treatments-friday-25-november}{Units
  of dental activity (UDA) changes to Band 2 treatments from Friday 25
  November}
\end{itemize}
\end{frame}

\begin{frame}{Banded Courses of Treatment as \% of February 2020
delivery (including FD)}
\protect\hypertarget{banded-courses-of-treatment-as-of-february-2020-delivery-including-fd}{}
\begin{center}\includegraphics{4-2025-Dental-Pack-National-reporting-up-to-end-of-March-2025_files/figure-beamer/banded CoT 2-1} \end{center}

\begin{itemize}
\tightlist
\item
  This graph shows time series on the the number of completed Courses of
  Treatment (CoTs) by band compared to pre-Covid delivery.
\end{itemize}
\end{frame}

\hypertarget{dental-care-practitioners-dcps-information}{%
\section{Dental Care Practitioners (DCPs)
Information}\label{dental-care-practitioners-dcps-information}}

\begin{frame}{Percentage of total Courses of Treatment (CoTs) delivered
which had DCP assistance}
\protect\hypertarget{percentage-of-total-courses-of-treatment-cots-delivered-which-had-dcp-assistance}{}
\includegraphics{4-2025-Dental-Pack-National-reporting-up-to-end-of-March-2025_files/figure-beamer/DCP1-1.pdf}

\begin{itemize}
\tightlist
\item
  This graph shows the percentage of CoTs carried out which benefited
  from involvement by Dental Care Practitioners.
\item
  DCP assistance data has been collected since October 2022 and DCP lead
  data collected since April 2024 so will likely take time for providers
  to improve reporting on it.
\end{itemize}
\end{frame}

\begin{frame}{Percentage of total UDAs delivered which had DCP
assistance by band}
\protect\hypertarget{percentage-of-total-udas-delivered-which-had-dcp-assistance-by-band}{}
\includegraphics{4-2025-Dental-Pack-National-reporting-up-to-end-of-March-2025_files/figure-beamer/DCP2-1.pdf}

\begin{itemize}
\tightlist
\item
  This chart shows the percentage of UDAs carried out which benefited
  from involvement by Dental Care Practitioners by treatment band.
\item
  DCP assistance data has been collected since October 2022 and DCP lead
  data collected since April 2024 so will likely take time for providers
  to improve reporting on it.
\item
  DCP lead data is only shown for Therapists as they deliver the
  majority of DCP-led activity.
\end{itemize}
\end{frame}

\hypertarget{patients-seen}{%
\section{Patients seen}\label{patients-seen}}

\begin{frame}{Unique dental patients seen}
\protect\hypertarget{unique-dental-patients-seen}{}
\begin{center}\includegraphics{4-2025-Dental-Pack-National-reporting-up-to-end-of-March-2025_files/figure-beamer/unique patients-1} \end{center}

\begin{itemize}
\tightlist
\item
  This chart shows the number of unique patients seen in a 24 month
  period for adults, and 12 month period for children.
\item
  A person seen more than once in the rolling 12/24 months would be
  counted only once.
\item
  NICE guidelines recommend that children should be seen at least 12
  months and adults at least every 24 months.
  \url{https://www.nice.org.uk/guidance/cg19/chapter/Recommendations}
\end{itemize}
\end{frame}

\begin{frame}{Unique dental patients seen as a percentage of February
2020 figures}
\protect\hypertarget{unique-dental-patients-seen-as-a-percentage-of-february-2020-figures}{}
\begin{center}\includegraphics{4-2025-Dental-Pack-National-reporting-up-to-end-of-March-2025_files/figure-beamer/unique patients percentage-1} \end{center}

\begin{itemize}
\tightlist
\item
  This chart shows the number of unique patients seen in a 24 month
  period for adults, and 12 month period for children compared to
  February 2020 levels.
\item
  A person seen more than once in the rolling 12/24 months would be
  counted only once.
\item
  NICE guidelines recommend that children should be seen at least 12
  months and adults at least every 24
  months.https://www.nice.org.uk/guidance/cg19/chapter/Recommendations
\end{itemize}
\end{frame}

\begin{frame}{Patients seen under the New Patient Premium definition}
\protect\hypertarget{patients-seen-under-the-new-patient-premium-definition}{}
New Patient Premium metrics have been removed from this pack as the
scheme has ended
\end{frame}

\begin{frame}{Patients seen under the New Patient Premium definition by
band}
\protect\hypertarget{patients-seen-under-the-new-patient-premium-definition-by-band}{}
New Patient Premium metrics have been removed from this pack as the
scheme has ended
\end{frame}

\begin{frame}{Eligible contracts that have seen new patients}
\protect\hypertarget{eligible-contracts-that-have-seen-new-patients}{}
New Patient Premium metrics have been removed from this pack as the
scheme has ended
\end{frame}

\begin{frame}{NPP activity to date against previous year's delivery}
\protect\hypertarget{npp-activity-to-date-against-previous-years-delivery}{}
This slide is not currently shown as the methodology is being reviewed.
The data is available in the accompanying excel file.
\end{frame}

\hypertarget{oral-health}{%
\section{Oral Health}\label{oral-health}}

\begin{frame}{Oral health risk assessment}
\protect\hypertarget{oral-health-risk-assessment}{}
\includegraphics{4-2025-Dental-Pack-National-reporting-up-to-end-of-March-2025_files/figure-beamer/BPE boxplot-revert-1.pdf}

\begin{itemize}
\tightlist
\item
  This chart shows the percentage of contracts who assign recall
  intervals of less than a year to 50\% or more of their low risk
  patients.
\item
  Low risk patients are adult patients receiving routine care with no
  evidence of decay and a Basic Periodontal Examination (PBE) score of 0
  or 1.
\end{itemize}
\end{frame}

\hypertarget{units-of-orthodontic-activity-uoa-information}{%
\section{Units of Orthodontic Activity (UOA)
Information}\label{units-of-orthodontic-activity-uoa-information}}

\begin{frame}{Units of Orthodontic Activity - Calendar activity data
standardised by working days}
\protect\hypertarget{units-of-orthodontic-activity---calendar-activity-data-standardised-by-working-days}{}
\begin{center}\includegraphics{4-2025-Dental-Pack-National-reporting-up-to-end-of-March-2025_files/figure-beamer/UOA_activity_Calendar1-1} \end{center}

\begin{itemize}
\tightlist
\item
  These graphs show the percentage of UOAs delivered compared to the
  expected delivery. Expected monthly delivery is standardised by
  working days in the month and calculated as Target UOAs*Monthly
  working days/working days per year.
\item
  For the current financial year, monthly performance is shown.
\item
  Over the course of the financial year this graph allows us to see
  patterns of delivery.
\end{itemize}
\end{frame}

\begin{frame}{Units of Orthodontic Activity - Calendar year to date
delivery}
\protect\hypertarget{units-of-orthodontic-activity---calendar-year-to-date-delivery}{}
\begin{center}\includegraphics{4-2025-Dental-Pack-National-reporting-up-to-end-of-March-2025_files/figure-beamer/YTD UOA-1} \end{center}

\begin{itemize}
\tightlist
\item
  This graph shows the progress towards delivering against contracted
  UOAs in the financial year. Here, expected delivery assumes delivery
  is equal across all months.
\end{itemize}
\end{frame}

\begin{frame}{Units of Orthodontic Activity started/completed}
\protect\hypertarget{units-of-orthodontic-activity-startedcompleted}{}
\begin{center}\includegraphics{4-2025-Dental-Pack-National-reporting-up-to-end-of-March-2025_files/figure-beamer/YTD UOA completed started-1} \end{center}

\begin{itemize}
\tightlist
\item
  This graph shows the number of courses of orthodontic treatment
  started and completed in a month plus the UOAs delivered.
\item
  FP17O forms are to be submitted at the start of orthodontic treatment
  and updated on completion. currently about 70\% are updated with a
  completion date.
\end{itemize}
\end{frame}

\hypertarget{information}{%
\section{111 Information}\label{information}}

\begin{frame}{111 triage dental related call volumes}
\protect\hypertarget{triage-dental-related-call-volumes}{}
\includegraphics{4-2025-Dental-Pack-National-reporting-up-to-end-of-March-2025_files/figure-beamer/111 referrals-1.pdf}

\begin{itemize}
\tightlist
\item
  This chart shows the demand for dental services via calls to 111.
\end{itemize}
\end{frame}

\hypertarget{appendix}{%
\section{Appendix}\label{appendix}}

\begin{frame}{Description of terms used in this pack}
\protect\hypertarget{description-of-terms-used-in-this-pack}{}
\begin{itemize}
\tightlist
\item
  \textbf{Calendar data} - This represents the month that a Course of
  Treatment (CoT) was completed.
\item
  \textbf{Scheduled data} - This represents the month that a Course of
  Treatment (CoT) was claimed for in the BSA Compass system.
\item
  \textbf{Course of Treatment (CoT)} - During a checkup, the dentist
  will identify any problems that need treatment. After discussing them
  with the patient, a Course of Treatment is created that deals with all
  of the problems.
\item
  \textbf{Unit of Dental Activity (UDA)} - Under the GDS contract,
  payments for primary care dentistry are made for units of dental
  activity (UDAs).
  \url{https://faq.nhsbsa.nhs.uk/knowledgebase/article/KA-01976}
\item
  \textbf{Treatment bands} - Treatment is grouped into bands which have
  UDAs associated with them. A brief overview is listed here

  \begin{itemize}
  \tightlist
  \item
    \textbf{Band 1} - check up and simple treatment e.g.~examination,
    x-rays and prevention advice (1 UDA)
  \item
    \textbf{Band 2} - mid range treatments e.g.~fillings, extractions,
    and root canal work plus Band 1 work (3-7 UDAs)
  \item
    \textbf{Band 3} - includes complex treatments e.g.~crowns, dentures,
    and bridges plus Band 1 and Band 2 work (12 UDAs)
  \item
    \textbf{Urgent} - a specified set of treatments delivered where oral
    health is likely to deteriorate significantly/ the person is in
    sever pain (1.2 UDAs)
  \item
    \textbf{Other} - mixed activity (0.75-1.2 UDAs)
  \end{itemize}
\item
  \textbf{Contracted UDAs} - Agreed number of UDAs a provider will
  deliver in a financial year
\item
  \textbf{Delivered UDAs} - Actual number of UDAs a provider has
  delivered
\item
  \textbf{Foundation Dentists (FD)} - Foundation Dentists are newly
  qualified dentists who are undertaking additional training in a Dental
  Practice. They are excluded from metrics unless explicitly stated.
\item
  \textbf{FP17} - The form (electronic) Dental practices use to submit
  claims for finished courses of treatment to BSA via Compass. Data from
  these forms is used in these slides.
  \url{https://www.nhsbsa.nhs.uk/activity-payment-and-pension-services/dental-activity-processing}
\end{itemize}

Helpful reading
\url{https://www.kingsfund.org.uk/insight-and-analysis/long-reads/dentistry-england-explained}
\end{frame}

\hypertarget{data-sources}{%
\section{Data sources}\label{data-sources}}

\begin{frame}{Data sources}
\protect\hypertarget{data-sources-1}{}
\begin{itemize}
\tightlist
\item
  BSA calendar data received from BSA and added to NCDR
\item
  Tables are:

  \begin{itemize}
  \tightlist
  \item
    {[}NHSE\_Sandbox\_PrimaryCareNHSContracts{]}.\protect\hyperlink{dental}{Dental}.{[}Calendar\_Contracts{]}
  \item
    {[}NHSE\_Sandbox\_PrimaryCareNHSContracts{]}.\protect\hyperlink{dental}{Dental}.{[}Calendar\_UDA\_Activity{]}
  \item
    {[}NHSE\_Sandbox\_PrimaryCareNHSContracts{]}.\protect\hyperlink{dental}{Dental}.{[}Calendar\_UDA\_Activity\_FD\_only{]}
  \item
    {[}NHSE\_Sandbox\_PrimaryCareNHSContracts{]}.\protect\hyperlink{dental}{Dental}.{[}Calendar\_Unique\_rolling{]}
  \item
    {[}NHSE\_Sandbox\_PrimaryCareNHSContracts{]}.\protect\hyperlink{dental}{Dental}.{[}Calendar\_DCP{]}
  \item
    {[}NHSE\_Sandbox\_PrimaryCareNHSContracts{]}.\protect\hyperlink{dental}{Dental}.{[}Calendar\_NPP\_Eligible\_Activity{]}
  \item
    {[}NHSE\_Sandbox\_PrimaryCareNHSContracts{]}.\protect\hyperlink{dental}{Dental}.{[}Calendar\_UOA\_Activity{]}
  \item
    {[}NHSE\_Sandbox\_PrimaryCareNHSContracts{]}.\protect\hyperlink{dental}{Dental}/{[}Calendar\_BPE{]}
  \end{itemize}
\item
  R code used to create this pack is available on Github

  \begin{itemize}
  \tightlist
  \item
    \url{https://github.com/nhsengland/SMT-Dental-Pack-PhODS}
  \end{itemize}
\item
  Data used in this pack is published on FutureNHS along with national
  and regional packs

  \begin{itemize}
  \tightlist
  \item
    \url{https://future.nhs.uk/DENTISTRY/view?objectID=39246768}
  \end{itemize}
\end{itemize}
\end{frame}

\end{document}
