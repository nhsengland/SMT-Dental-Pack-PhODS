% Options for packages loaded elsewhere
\PassOptionsToPackage{unicode}{hyperref}
\PassOptionsToPackage{hyphens}{url}
%
\documentclass[
  8pt,
  ignorenonframetext,
  aspectratio = 169]{beamer}
\usepackage{pgfpages}
\setbeamertemplate{caption}[numbered]
\setbeamertemplate{caption label separator}{: }
\setbeamercolor{caption name}{fg=normal text.fg}
\beamertemplatenavigationsymbolsempty
% Prevent slide breaks in the middle of a paragraph
\widowpenalties 1 10000
\raggedbottom
\setbeamertemplate{part page}{
  \centering
  \begin{beamercolorbox}[sep=16pt,center]{part title}
    \usebeamerfont{part title}\insertpart\par
  \end{beamercolorbox}
}
\setbeamertemplate{section page}{
  \centering
  \begin{beamercolorbox}[sep=12pt,center]{part title}
    \usebeamerfont{section title}\insertsection\par
  \end{beamercolorbox}
}
\setbeamertemplate{subsection page}{
  \centering
  \begin{beamercolorbox}[sep=8pt,center]{part title}
    \usebeamerfont{subsection title}\insertsubsection\par
  \end{beamercolorbox}
}
\AtBeginPart{
  \frame{\partpage}
}
\AtBeginSection{
  \ifbibliography
  \else
    \frame{\sectionpage}
  \fi
}
\AtBeginSubsection{
  \frame{\subsectionpage}
}
\usepackage{amsmath,amssymb}
\usepackage{lmodern}
\usepackage{iftex}
\ifPDFTeX
  \usepackage[T1]{fontenc}
  \usepackage[utf8]{inputenc}
  \usepackage{textcomp} % provide euro and other symbols
\else % if luatex or xetex
  \usepackage{unicode-math}
  \defaultfontfeatures{Scale=MatchLowercase}
  \defaultfontfeatures[\rmfamily]{Ligatures=TeX,Scale=1}
  \setsansfont[]{Calibri Light}
\fi
% Use upquote if available, for straight quotes in verbatim environments
\IfFileExists{upquote.sty}{\usepackage{upquote}}{}
\IfFileExists{microtype.sty}{% use microtype if available
  \usepackage[]{microtype}
  \UseMicrotypeSet[protrusion]{basicmath} % disable protrusion for tt fonts
}{}
\makeatletter
\@ifundefined{KOMAClassName}{% if non-KOMA class
  \IfFileExists{parskip.sty}{%
    \usepackage{parskip}
  }{% else
    \setlength{\parindent}{0pt}
    \setlength{\parskip}{6pt plus 2pt minus 1pt}}
}{% if KOMA class
  \KOMAoptions{parskip=half}}
\makeatother
\usepackage{xcolor}
\newif\ifbibliography
\usepackage{graphicx}
\makeatletter
\def\maxwidth{\ifdim\Gin@nat@width>\linewidth\linewidth\else\Gin@nat@width\fi}
\def\maxheight{\ifdim\Gin@nat@height>\textheight\textheight\else\Gin@nat@height\fi}
\makeatother
% Scale images if necessary, so that they will not overflow the page
% margins by default, and it is still possible to overwrite the defaults
% using explicit options in \includegraphics[width, height, ...]{}
\setkeys{Gin}{width=\maxwidth,height=\maxheight,keepaspectratio}
% Set default figure placement to htbp
\makeatletter
\def\fps@figure{htbp}
\makeatother
\setlength{\emergencystretch}{3em} % prevent overfull lines
\providecommand{\tightlist}{%
  \setlength{\itemsep}{0pt}\setlength{\parskip}{0pt}}
\setcounter{secnumdepth}{-\maxdimen} % remove section numbering
\titlegraphic{\includegraphics[width=12cm, height=2cm]{footer.JPG}}
\setbeamertemplate{footline}
{
    \leavevmode%
    \hbox{%
        \begin{beamercolorbox}[wd = .5\paperwidth, ht = 1ex, dp = 1ex, center]{author in head/foot}%
            CONFIDENTIAL: Not for onward sharing
        \end{beamercolorbox}%
        \begin{beamercolorbox}[wd = .5\paperwidth, ht = 1ex, dp = 1ex, center]{date in head/foot}%
            \insertframenumber{}
        \end{beamercolorbox}
    }%
    \vskip3pt%
}

\setbeamertemplate{headline}
{
    \leavevmode%
    \hbox{%
        \begin{beamercolorbox}[wd = \paperwidth, ht = 20pt, dp = 0pt, right]{date in head/foot}%
            \includegraphics[width=1.5cm, height=4.5cm]{logo.JPG}
        \end{beamercolorbox}
    }%
    % \vskip3pt%
}

\addtobeamertemplate{frametitle}{\vspace*{-0.7cm}}{\vspace*{0.2cm}}



% \usepackage{graphicx}
% \usepackage{fancyhdr}
%    \pagestyle{fancy}
%    \fancyhead[R]{\includegraphics[width=1.5cm, height=4.5cm]{logo.JPG}}
% \renewcommand{\headrulewidth}{0pt}
% 
% \addtobeamertemplate{footline}{\begin{center}\rule{0.6\paperwidth}{0.4pt}\end{center}\vspace*{-1ex}}{}

% \setbeamertemplate{headline}
% {
%     \leavevmode%
%     \hbox{%
%         \begin{beamercolorbox}[wd = .5\paperwidth, ht = 1ex, dp = 1ex, center]{author in head/foot}%
%             \pgfdeclareimage[height=1.5cm, width=4.5cm]{logo}{logo.JPG}
%             \logo{\pgfuseimage{logo}}
%         \end{beamercolorbox}
%     }%
%     \vskip3pt%
% }
% 
% \addtobeamertemplate{headline}{\begin{center}\rule{0.6\paperwidth}{0.4pt}\end{center}\vspace*{-1ex}}{}

\ifLuaTeX
  \usepackage{selnolig}  % disable illegal ligatures
\fi
\IfFileExists{bookmark.sty}{\usepackage{bookmark}}{\usepackage{hyperref}}
\IfFileExists{xurl.sty}{\usepackage{xurl}}{} % add URL line breaks if available
\urlstyle{same} % disable monospaced font for URLs
\hypersetup{
  pdftitle={Dental Data Pack},
  pdfauthor={July, 2024 - Reporting up to end of June, 2024},
  hidelinks,
  pdfcreator={LaTeX via pandoc}}

\title{Dental Data Pack}
\subtitle{National}
\author{July, 2024 - Reporting up to end of June, 2024}
\date{}

\begin{document}
\frame{\titlepage}

\begin{frame}[allowframebreaks]
  \tableofcontents[hideallsubsections]
\end{frame}
\hypertarget{introduction}{%
\section{Introduction}\label{introduction}}

\begin{frame}{Introduction}
\protect\hypertarget{introduction-1}{}
This slide pack has been streamlined to help users to easily access the
most important information. This change took effect as of June 2024
data. As part of this change we have also sought to make the slides more
useful by making the following changes.

\begin{itemize}
\tightlist
\item
  Swapping from scheduled data (the month the dentist claimed for
  payment following a finished course of treatment) to Calendar month
  data (the month the course of treatment finished). This change better
  reflects the activity of dentists.
\item
  Including only GDS/PDS/PDS+ contracts where Total contracted UDAs
  \textgreater100. This means contracts with low UDA targets are
  excluded since they likely do not represent \textbf{Caroline to supply
  some wording}
\item
  A national HTML will no longer be produced, instead the data is an
  excel file available alongside this pack.
\end{itemize}

\begin{block}{Metrics and outputs soon to be added}
\protect\hypertarget{metrics-and-outputs-soon-to-be-added}{}
\begin{itemize}
\tightlist
\item
  Orthodontics metrics
\item
  Oral health risk assessment
\item
  New Patient Premiums metrics
\item
  ICB level HTML packs will be available in the future, for now the data
  is available in the excel document.
\end{itemize}
\end{block}
\end{frame}

\begin{frame}{Calendar Vs scheduled data}
\protect\hypertarget{calendar-vs-scheduled-data}{}
Analyses in this report now only use \textbf{calendar data} Previous
packs used scheduled data. An explanation on the difference is given
below. The appendix also provides more detail on how the switch effects
the data.

\begin{block}{Calendar data explanation}
\protect\hypertarget{calendar-data-explanation}{}
\begin{itemize}
\tightlist
\item
  Calendar data represents the month that a Course of Treatment (COT)
  was completed.
\item
  For calendar data, if a COT was completed in February but not declared
  till March, that activity would still be recorded as occurring in
  February.
\end{itemize}
\end{block}

\begin{block}{Scheduled data explanation}
\protect\hypertarget{scheduled-data-explanation}{}
\begin{itemize}
\tightlist
\item
  Scheduled data represents the month that a Course of Treatment (COT)
  was claimed for in the BSA COMPASS system.
\item
  For scheduled data, if a COT was completed in February but not
  declared till March, the financial activity would recorded as
  occurring in March.
\end{itemize}

The swap to using calendar data allows us to more accurately see what
activity is happening per month rather than what payments are being
claimed per month.

Note that following a finished Course of Treatment (COT), dentists have
62 days to declare their activity via FP17 forms to the BSA COMPASS
system. This means when using calendar data, the most recent two months
data are incomplete and subject to change, particularly for the most
recent month. We expect final data to be higher.
\end{block}
\end{frame}

\hypertarget{units-of-dental-activity-uda-information}{%
\section{Units of Dental Activity (UDA)
Information}\label{units-of-dental-activity-uda-information}}

\begin{frame}{Units of Dental Activity - Calendar activity data
standardised by working days}
\protect\hypertarget{units-of-dental-activity---calendar-activity-data-standardised-by-working-days}{}
\begin{center}\includegraphics{C:\Users\LXu\Documents\SMT-Dental-Pack-PhODS\reports\SMT Dental Pack Monthly - National 2024-07-08_files/figure-beamer/UDA_activity_Calendar2-1} \end{center}

\begin{center}\includegraphics{C:\Users\LXu\Documents\SMT-Dental-Pack-PhODS\reports\SMT Dental Pack Monthly - National 2024-07-08_files/figure-beamer/UDA_activity_Calendar2-2} \end{center}

\begin{itemize}
\tightlist
\item
  These graphs show, the percentage of UDAs delivered compared to the
  expected delivery. Expected monthly delivery is standardised by
  working days in the month and calculated as Target UDAs* Monthly
  working days/working days per year.
\item
  For the current financial year, monthly performance is shown. For the
  previous financial years, annual average performance is shown.
\item
  Over the course of the financial year this graph allows us to see
  patterns of delivery.
\end{itemize}
\end{frame}

\begin{frame}{Units of Dental Activity - Calendar year to date delivery}
\protect\hypertarget{units-of-dental-activity---calendar-year-to-date-delivery}{}
\begin{center}\includegraphics{C:\Users\LXu\Documents\SMT-Dental-Pack-PhODS\reports\SMT Dental Pack Monthly - National 2024-07-08_files/figure-beamer/YTD UDA-1} \end{center}

\begin{itemize}
\tightlist
\item
  This graph shows the progress towards delivering against contracted
  UDAs in the financial year. Here, expected delivery assumes delivery
  is equal across all months.
\item
  Providers are expected to have achieved 30\% of their annual delivery
  by the financial year mid point (October)
\item
  Implications of not reaching target delivery at the end of the
  financial year can be seen via this link:
  \url{https://faq.nhsbsa.nhs.uk/knowledgebase/article/KA-01672/en-us}
\end{itemize}
\end{frame}

\begin{frame}{Banded courses of treatment standardised by working days
(included FD)}
\protect\hypertarget{banded-courses-of-treatment-standardised-by-working-days-included-fd}{}
\begin{center}\includegraphics{C:\Users\LXu\Documents\SMT-Dental-Pack-PhODS\reports\SMT Dental Pack Monthly - National 2024-07-08_files/figure-beamer/unnamed-chunk-1-1} \end{center}

\begin{itemize}
\tightlist
\item
  This graph shows the number of completed Courses of Treatment (CoTs)
  by band over time. Data has been standardised to working days to
  remove variation due to month length. It is calculated as UDAs*Monthly
  working days/working days per year.
\end{itemize}
\end{frame}

\begin{frame}{Banded Courses of Treatment as \% of February 2020
delivery (included FD)}
\protect\hypertarget{banded-courses-of-treatment-as-of-february-2020-delivery-included-fd}{}
\begin{center}\includegraphics{C:\Users\LXu\Documents\SMT-Dental-Pack-PhODS\reports\SMT Dental Pack Monthly - National 2024-07-08_files/figure-beamer/unnamed-chunk-2-1} \end{center}

\begin{itemize}
\tightlist
\item
  This graph shows time series on the the number of completed Courses of
  Treatment (CoTs) by band compared to pre-covid delivery.
\end{itemize}
\end{frame}

\hypertarget{dental-care-practitioners-dcps-information}{%
\section{Dental Care Practitioners (DCPs)
Information}\label{dental-care-practitioners-dcps-information}}

\begin{frame}{Percentage of total Courses of Treatment (CoTs) delivered
which had DCP assistance}
\protect\hypertarget{percentage-of-total-courses-of-treatment-cots-delivered-which-had-dcp-assistance}{}
\includegraphics{C:/Users/LXu/Documents/SMT-Dental-Pack-PhODS/reports/SMT Dental Pack Monthly - National 2024-07-08_files/figure-beamer/unnamed-chunk-3-1.pdf}

\begin{itemize}
\tightlist
\item
  This graph shows the percentage of UDAs carried out which benefited
  from assistance by Dental Care Practitioners.DCP data has been
  collected since October 2022 so will likely take time for providers to
  improve reporting on it.
\end{itemize}
\end{frame}

\begin{frame}{Percentage of total UDAs delivered which had DCP
assistance by band}
\protect\hypertarget{percentage-of-total-udas-delivered-which-had-dcp-assistance-by-band}{}
\includegraphics{C:/Users/LXu/Documents/SMT-Dental-Pack-PhODS/reports/SMT Dental Pack Monthly - National 2024-07-08_files/figure-beamer/unnamed-chunk-4-1.pdf}

\begin{itemize}
\tightlist
\item
  This chart shows the percentage of UDAs carried out which benefited
  from assistance by Dental Care Practitioners by treatment band. DCP
  data has been collected since October 2022 so will likely take time
  for providers to improve reporting on it.
\end{itemize}
\end{frame}

\hypertarget{unique-patients}{%
\section{Unique patients}\label{unique-patients}}

\begin{frame}{Unique dental patients seen}
\protect\hypertarget{unique-dental-patients-seen}{}
\begin{center}\includegraphics{C:\Users\LXu\Documents\SMT-Dental-Pack-PhODS\reports\SMT Dental Pack Monthly - National 2024-07-08_files/figure-beamer/unique patients-1} \end{center}

\begin{itemize}
\tightlist
\item
  This chart shows the number of unique patients seen in a 24 month
  period for adults, and 12 month period for children.
\item
  A person seen more than once in the rolling 12/24 months would be
  counted only once
\item
  NICE guidelines recommend that children should be seen at least 12
  months and adults at least every 24 months.
  \url{https://www.nice.org.uk/guidance/cg19/chapter/Recommendations}
\end{itemize}
\end{frame}

\begin{frame}{Unique dental patients seen as a percentage of February
2020 figures}
\protect\hypertarget{unique-dental-patients-seen-as-a-percentage-of-february-2020-figures}{}
\begin{center}\includegraphics{C:\Users\LXu\Documents\SMT-Dental-Pack-PhODS\reports\SMT Dental Pack Monthly - National 2024-07-08_files/figure-beamer/unique patients percentage-1} \end{center}

\begin{itemize}
\tightlist
\item
  This chart shows the number of unique patients seen in a 24 month
  period for adults, and 12 month period for children compared to
  February 2020 levels
\item
  A person seen more than once in the rolling 12/24 months would be
  counted only once
\item
  NICE guidelines recommend that children should be seen at least 12
  months and adults at least every 24 months.
  \url{https://www.nice.org.uk/guidance/cg19/chapter/Recommendations}
\end{itemize}
\end{frame}

\hypertarget{units-of-orthadontic-activity-uoa-information}{%
\section{Units of Orthadontic Activity (UOA)
Information}\label{units-of-orthadontic-activity-uoa-information}}

\begin{frame}{Units of Orthadontic Activity (UOA) Information}
\begin{itemize}
\tightlist
\item
  UOA data is not yet available in calendar format
\end{itemize}
\end{frame}

\hypertarget{information}{%
\section{111 Information}\label{information}}

\begin{frame}{111 triage dental related call volumes}
\protect\hypertarget{triage-dental-related-call-volumes}{}
\includegraphics{C:/Users/LXu/Documents/SMT-Dental-Pack-PhODS/reports/SMT Dental Pack Monthly - National 2024-07-08_files/figure-beamer/unnamed-chunk-5-1.pdf}
* This chart shows the demand for dental services via calls to 111.
\end{frame}

\hypertarget{annex}{%
\section{Annex}\label{annex}}

\begin{frame}{Annex of terms for dental data}
\protect\hypertarget{annex-of-terms-for-dental-data}{}
\begin{itemize}
\tightlist
\item
  \textbf{Calendar data} - This represents the date that a Course of
  Treatment (COT) was completed.
\item
  \textbf{Scheduled data} - This represents the date that a Course of
  Treatment (COT) was claimed for in the BSA COMPASS system.
\item
  \textbf{Course of Treatment (COT)} - During a checkup, the dentist
  will identify any problems that need treatment. After discussing them
  with the patient, a Course of Treatment is created that deals with all
  of the problems.
\item
  \textbf{Unit of Dental Activity (UDA)} - Under the GDS contract,
  payments for primary care dentistry are made for units of dental
  activity (UDAs).
  \url{https://faq.nhsbsa.nhs.uk/knowledgebase/article/KA-01976}
\item
  \textbf{Treatment bands} - Treatment is grouped into bands which have
  UDAs associated with them. A brief overview is listed here

  \begin{itemize}
  \tightlist
  \item
    \textbf{Band 1} - check up and simple treatment e.g.~examination,
    x-rays and prevention advice (1 UDA)
  \item
    \textbf{Band 2} - mid range treatments e.g.~fillings, extractions,
    and root canal work plus Band 1 work (3-7 UDAs)
  \item
    \textbf{Band 3} - includes complex treatments e.g.~crowns, dentures,
    and bridges plus Band 1 and Band 2 work (12 UDAs)
  \item
    \textbf{Urgent} - a specified set of treatments delivered where oral
    health is likely to deteriorate significantly/ the person is in
    sever pain (1.2 UDAs)
  \item
    \textbf{Other} -* \textbf{not sure - need to find out} (mixed UDAs)
  \end{itemize}
\item
  \textbf{Contracted UDAs} - Agreed number of UDAs a provider will
  deliver in a financial year
\item
  \textbf{Delivered UDAs} - Actual number of UDAs a provider has
  delivered
\item
  \textbf{Foundation Dentists (FD)} - Foundation Dentists are newly
  qualified dentists who are undertaking additional training in a Dental
  Practice. They are excluded from metrics unless explicitly included.
\item
  \textbf{FP17} - The Form (electronic) Dental practices use to submit
  claims for finished courses of treatment to BSA via COMPASS. Data from
  these forms is used in these slides.
  \url{https://www.nhsbsa.nhs.uk/activity-payment-and-pension-services/dental-activity-processing}
\end{itemize}

Kings fund -
\url{https://www.kingsfund.org.uk/insight-and-analysis/long-reads/dentistry-england-explained}
\end{frame}

\begin{frame}{Impact of swapping to calendar data}
\protect\hypertarget{impact-of-swapping-to-calendar-data}{}
\textbf{Clare and Jenny to add}
\end{frame}

\end{document}
